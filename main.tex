\documentclass[12pt]{article}
\usepackage{arabic_arxiv_2024}
\usepackage{url}

% \title{\textarabic{عنوان البحث}}
% \author{\textarabic{المؤلف الأول} \thanks{\textarabic{ملاحظة حول المؤلف الأول}} \\ \textit{الجامعة أو المؤسسة}
%    \and \textarabic{المؤلف الثاني} \\ \textit{الجامعة أو المؤسسة}
%    \and \textarabic{المؤلف الثاني} \\ \textit{الجامعة أو المؤسسة}}
\title{عنوان البحث}
\author[1]{اسم ولقب المؤلف الأول \thanks{عنوان البريد الإلكتروني}}
\author[2]{اسم ولقب المؤلف الثاني}
\author[1,3]{اسم ولقب المؤلف الثالث}

\affil[1]{
الرتبة الأكاديمية للمؤلف، المجموعة التعليمية أو الوحدة التنظيمية ذات الصلة، اسم المنظمة، البلد
}
\affil[2]{
الرتبة الأكاديمية للمؤلف، المجموعة التعليمية أو الوحدة التنظيمية ذات الصلة، اسم المنظمة، البلد
}
\affil[3]{
الرتبة الأكاديمية للمؤلف، المجموعة التعليمية أو الوحدة التنظيمية ذات الصلة، اسم المنظمة، البلد
}

\translator{ اسم ولقب المترجم, الرتبة الأكاديمية، اسم المنظمة، البلد \thanks{عنوان البريد الإلكتروني}}

\date{}
\begin{document}

\maketitle

\begin{abstract}
يهدف هذا الشرح لتفصيل كيفية تنسيق مقال وأقسامه المختلفة ليكون جاهزا للنشر في أرشيف العرب. يتم تعريف أجزاء المقال المختلفة مسبقًا، بما في ذلك العنوان والمؤلفين والملخص والنص...إلخ.
إذا لم يكن المقال مترجماً، فيجب إعداد ملخص المقال المقترح في فقرة أو فقرتين بمعدل 250 كلمة كحد أقصى.
 يجب أن يبين الملخص موضوع البحث ونتائجه بشكل واضح وصريح. وهذا يعني الإشارة إلى ما تم القيام به، وكيف، ولأي غرض، وما هي النتائج التي تم الحصول عليها. و يجدر تجنب ذكر تفاصيل العمل والأشكال والجداول والصيغ والمراجع في الملخص.
\end{abstract}

\begin{keywords}
اختر ما يصل إلى 5 كلمات ككلمات رئيسية.
\end{keywords}


\section{تقديم الأوراق لأرشيف العرب}

يرجى قراءة التعليمات أدناه بعناية واتباعها بدقة.

\subsection{النموذج}

يجب أن تكون الأوراق المُقدمة إلى أرشيف العرب معدة وفقًا للإرشادات الموضحة هنا. لا يوجد أي قيود على عدد صفحات المقال. إذا كان المقال المقدم قد سبق نشره باللغة العربية، فلا بأس بتقديمه مباشرة دون استعمال هذا النموذج.  


يُطلب من المؤلفين استخدام ملفات نمط أرشيف العرب \LaTeX{} المتاحة على موقع أرشيف العرب كما هو مشار إليه أدناه. يرجى التأكد من استخدام النسخ الحالية من الملفات وليس الإصدارات السابقة. يمكن أن يكون تعديل ملفات النمط أسبابًا للرفض.

\subsection{تحميل النموذج}

ملفات النموذج الخاص بأرشيف العرب موجودة على موقع الويب التالي:

\begin{center}
\url{http://www.arabarxiv.org/}
\end{center}

الملف \verb+arabarxiv_2024.pdf+ يحتوي على هذه التعليمات ويوضح متطلبات التنسيق المختلفة التي يجب أن تلبيها ورقتك المقدمة إلى أرشيف العرب.


الملف النمطي \XeLaTeX{} الوحيد المدعوم لأرشيف العرب 2024 هو \verb+arabarxiv_2024.sty+، تم إعادة كتابته لـ \XeLaTeX{}.

\section{تقسيم المقال}
يجب أن تتضمن كل مقالة هذه الأقسام الرئيسية: الملخص، والكلمات الدالة، والمقدمة، والمحتوى الرئيسي، والخاتمة، والمراجع. الأقسام الأخرى مثل الشكر والتقدير والعناوين الفرعية اختيارية وتوضع قبل المراجع باستثناء الملحقات .(Appendix)

يحتوي كل قسم أو قسم فرعي على فقرة واحدة أو أكثر (فقرات).

\subsection{مميزات العنوان ومؤلفي المقال}
يجب أن يكون عنوان المقال قصيرًا، دالا على العمل البحثي، و أن لا يتجاوز السطرين.
بعد عنوان المقال يجب كتابة أسماء مؤلفي المقال. عند كتابة أسماء المؤلفين، تجنب ذكر الألقاب مثل الأستاذ، الطبيب، المهندس، إلخ. يمكن ذكر اسم الجامعة أو مكان عمل المؤلف مع العنوان وعنوان البريد الإلكتروني.

\subsection{الكلمات الدالة}
يجب أن تشير الكلمات الدالة إلى المواضيع الرئيسية والفرعية للمقالة. افصل بين كل كلمة رئيسية بفاصلة (،). ضع نقطة في نهاية الكلمة الرئيسية الأخيرة. الكلمات الأكثر صلة يجب كتابتها أولاً.

\subsection{الكلمات المختصرة}
لا يسمح باستخدام الكلمات المختصرة في الملخص أو في الكلمات المفتاحية، يجب ذكرها بشكلها الكامل و  كتابة الإختصار داخل زوج من الخطوط الهلالية (بين قوسين).
في النص يسمح باستعمال  الكلمات المختصرة على أن تبين كاملة مع ذكر اختصارها بين قوسين في أول موضع في النص، حتى لو تم تعريفها في الملخص أو الكلمات الدالة.

\subsection{المقالات المترجمة}
 يجب أن يحصل مترجمو الأبحاث على إذن من المؤلفين الأصليين قبل 
 ترجمة أعمالهم، إلا إذا كان العمل مفتوح الوصول ( بالإنجليزية access open ).
 في هذه الحالة، يتبع المترجم نفس عدد الكلمات المستخدمة في الملخص الأصلي أو ما يقرب من ذلك.
 إذا لم يكن المقال المترجم يحتوي على كلمات دالة،  فيرجى من المترجم إضافتها بما يتناسب مع محتوى المقال. 

%Hadjer changed after here%
\section{العناوين: المستوى الأول}
\label{headings}

يجب أن تكون جميع العناوين مُحاذاة إلى اليمين.

يجب أن تكون العناوين من المستوى الأول بحجم 14 نقطة.

\subsection{العناوين: المستوى الثاني}

يجب أن تكون العناوين من المستوى الثاني بحجم 12 نقاط.

\subsubsection{العناوين: المستوى الثالث}

يجب أن تكون العناوين من المستوى الثالث بحجم 12 نقاط.

\paragraph{الفقرات}

هناك أيضًا أمر \verb+\paragraph+ متاح، يعين العنوان على شكل عريض ومُحاذاة إلى اليمين ومضمن في النص مع مسافة 1,em تلي العنوان.

\section{الاقتباسات، الصور، الجداول، والملحقات}
\label{others}

هذه التعليمات تنطبق على الجميع.

\subsection{الاقتباسات في النص}

في قسم المراجع في هذا القالب، نتبع أسلوب الاقتباس المعتمد في معاملات الـ IEEE. يُسمح باستخدام الحروف اللاتينية أو العربية للإشارة إلى المراجع. نشجع بشدة على استخدام الاقتباسات الكاملة باللغة العربية لتحقيق التكامل والدقة في البحث. يتم تضمين النص المترجم في ملف BibTeX. يرجى الالتزام بالإرشادات التالية لضمان الاتساق والدقة في تنسيق المراجع:

\begin{enumerate}
  \item للإشارة إلى مصدر بالحروف اللاتينية، يُرجى اتباع أسلوب IEEE القياسي في الاقتباس.
  \item للإشارة إلى مصدر باللغة العربية، يُرجى استخدام نفس التنسيق مع تحويل النص إلى اللغة العربية بشكل كامل.
\end{enumerate}

يتم تحميل حزمة \verb|natbib| تلقائيًا للتعامل مع الاقتباسات. يمكنك الاقتباس باستخدام أسلوب المؤلف/السنة أو العدد، مع الحفاظ على الاتساق طوال النص. بالنسبة لتنسيق المراجع، يُقبل أي أسلوب طالما أنه مستخدم بشكل منتظم.

يمكن الاطلاع على توثيق حزمة \verb|natbib| عبر الرابط التالي:
\begin{center}
\url{http://mirrors.ctan.org/macros/latex/contrib/natbib/natnotes.pdf}
\end{center}


للاقتباس باستخدام رقم المرجع، استخدم \verb|\cite|، مثل:
\{\textarabic{السنبري2023التقييم}\}\verb|\cite|

يُنتج:
\begin{quote}
\cite{السنبري2023التقييم} قاموا بتحقيق\dots
\end{quote}

قد تسبب النسخة العربية ل \verb|BibTeX| بعض المشاكل، لتحويلها إلى التنسيق المناسب، يرجى الإطلاع على الرابط التالي: 
\url{https://arabarxiv.org/bibtex}

\subsection{الهوامش}
يجب استخدام الهوامش بحذر. إذا كنت بحاجة إلى حاشية، فقم بتوضيح الهوامش برقم\footnote{عينة من الهامش الأول.} في النص. ضع الهوامش في أسفل الصفحة التي تظهر فيها. يرجى ملاحظة أن الهوامش يجب أن تكون منسقة بشكل صحيح \emph{بعد} علامات الترقيم\footnote{كما في هذا المثال.}

بالإضافة إلى ذلك، يمكنك تضمين روابط للأكواد والأدوات بالأحرف اللاتينية في الهوامش. هذا يسهل الوصول إلى الموارد الإلكترونية ذات الصلة ويعزز فائدة الوثيقة.

\subsection{الصور}

\begin{figure}[ht]
\centering
\includegraphics[width=0.5\textwidth]{example-image}
\caption{\textarabic{تسمية الشكل}}
\end{figure}

يجب أن تكون جميع الرسومات نظيفة وواضحة. يجب أن تكون الخطوط داكنة بما يكفي لأغراض الاستنساخ. يظهر رقم الصورة والتوضيح دائمًا بعد الصورة. ضع مسافة سطر واحد قبل توضيح الصورة ومسافة سطر واحد بعد الصورة. يتم ترقيم الصور بتتابع.

يمكنك استخدام صور ملونة. ومع ذلك، من الأفضل أن تكون توضيحات الصور ونص الورقة قابلة للقراءة إذا تم طباعة الورقة سواء بالأبيض والأسود أو بالألوان. يُنصح بمراجعة قسم "كيفية إنشاء الرسومات باستخدام Python" على موقع الويب الخاص بنا للحصول على إرشادات وتقنيات مفصلة حول تصميم الرسومات الفعالة.


% الجداول
\subsection{الجداول}

يجب أن تكون جميع الجداول متوسطة ونظيفة وواضحة. يظهر رقم الجدول والعنوان دائمًا قبل الجدول. ضع مسافة سطر واحد قبل عنوان الجدول ومسافة سطر واحد بعد عنوان الجدول، ومسافة سطر واحد بعد الجدول.يتم ترقيم الجداول بتتابع.

لتصميم جداول أكثر تعقيدًا، يمكن استخدام حزمة \verb|multirow|، والتي تتيح دمج عدة صفوف في خلية واحدة، مما يعطي مرونة أكبر في تنظيم البيانات داخل الجدول.

 نوصي بشدة باستخدام حزمة \verb|booktabs| التي تسمح بطباعة جداول ذات جودة عالية واحترافية، مع توفير خطوط أفقية محسّنة لتقسيم البيانات بوضوح.

\begin{table}[ht]
\centering
\caption{\textarabic{تسمية الجدول}}\label{tab:table}
\begin{tabular}{|c|c|c|}
\hline
\textarabic{العمود 1} & \textarabic{العمود 2} & \textarabic{العمود 3} \\
\hline
\textarabic{بيانات} & \textarabic{بيانات} & \textarabic{بيانات} \\
\hline
\end{tabular}
\end{table}

% الرياضيات
\subsection{الرياضيات والنظريات}
في هذا القالب، يُسمح بكتابة الصيغ والمعادلات الرياضية باستخدام الحروف اللاتينية، وذلك نظرًا لطبيعتها العالمية واستخدام الرموز الرياضية المعترف بها دوليًا. ومع ذلك، نشجع على كتابة جميع الأجزاء الأخرى من الورقة البحثية، بما في ذلك النصوص الوصفية وبيانات النظريات واللمم، باللغة العربية لتعزيز استخدام اللغة العربية في الأبحاث العلمية. هذا النهج يدمج بين الدقة الرياضية وتعزيز اللغة العربية.

\begin{theorem}
لتكن \( f \) دالة مستمرة على الفترة \([a, b]\). ثم إذا كانت \( f(a) \) و \( f(b) \) تحمل إشارات متعاكسة، فهناك على الأقل نقطة \( c \) في \((a, b)\) حيث \( f(c) = 0 \).
\end{theorem} 

\begin{theorem}
لتكن \( \text{ف} \) دالة مستمرة على الفترة \([\text{أ}, \text{ب}]\). ثم إذا كانت \( \text{ف}(\text{أ}) \) و \( \text{ف}(\text{ب}) \) تحمل إشارات متعاكسة، فهناك على الأقل نقطة \( \text{ج} \) في \((\text{أ}, \text{ب})\) حيث \( \text{ف}(\text{ج}) = 0 \).
\end{theorem}

\subsection{\textarabic{الخوارزميات}}
لإنشاء خوارزميات باللغة العربية في LaTeX، يُنصح باستخدام حزمة algorithmicx. تتميز هذه الحزمة بتوافقها مع النصوص بكل من اللغة الإنجليزية والعربية، مما يسهل عملية تنسيق الخوارزميات بلغات متعددة. فيما يلي مثال توضيحي يبين كيفية استخدام الحزمة لكتابة خوارزمية باللغة العربية، مع التأكيد على البنية الأساسية والتعليقات:

\begin{algorithm}
\caption{خوارزمية مثال}
\begin{algorithmic}[1]
\Procedure{اسم الخوارزمية}{}
\بداية
  \من{i}{1}{n} \Comment{\textarabic{حلقة for}}
    \بداية
      \إذا{$i$ فردي}
        \بداية
          \وإلا
        \إذاكان
  \إلى
\EndProcedure
\end{algorithmic}
\end{algorithm}

\begin{ack}
استخدم عناوين المستوى الأول غير المرقمة للشكر. يتم وضع جميع عبارات الشكر في نهاية الورقة قبل قائمة المراجع. يُمكن إضافة الشكر الشخصي بشكل اختياري. علاوة على ذلك، يُلزم عليك الإعلان الإلزامي عن التمويل (الأنشطة المالية التي دعمت العمل المقدم) والمصالح المتعارضة (الأنشطة المالية ذات الصلة خارج العمل المقدم).
\end{ack}


\subsection{المعلومات الملحقة}

قد يرغب الكتّاب في تضمين معلومات إضافية اختيارية (أدلة كاملة، تجارب ورسوم بيانية إضافية) في الملحق. يجب أن تكون جميع المعلومات المقدمة في الملحق مرتبطة بشكل مباشر بالمحتوى الرئيسي للورقة البحثية. في حالات معينة، يُسمح باستخدام الوصف بالحروف اللاتينية في الملحق إذا كان ذلك ضروريًا للفهم الواضح للمادة. يتبع الملحق قسم المراجع مباشرةً، ويجب تنظيمه بطريقة تسهل الوصول إلى المعلومات وتفسيرها. يُشجع الكتّاب على استخدام هذا القسم لتوفير تعمق أكبر في البيانات والمنهجيات المستخدمة، وكذلك لعرض نتائج إضافية قد تكون مفيدة للقارئ.

\bibliographystyle{ieeetr}
\bibliography{references}
%https://www.overleaf.com/learn/latex/Questions/Which_BibTeX_Styles_are_Available_on_Overleaf%3F
\end{document}
